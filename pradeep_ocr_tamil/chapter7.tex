\chapter{Application and Conclusion}


\section{Application}
% \subsection{Parallel corpus}
OCR system is useful for creating volume of corpus by converting existing scanned document into 
plain text. It can also be used for creating encyclopedia from existing old documents. Since  
multilingual encyclopedia can serve as parallel corpus, OCR system would be a handy tool 
for creating dataset from existing paperback encyclopedia. Much of the WWW document images would be 
useful only if it can be indexed because only then it becomes searchable. So OCR plays a major 
role there too. OCR system can also be used in post office where handwritten address is converted into digital format for efficient storage and retrieval. Apart from these examples, OCR system can be 
used anywhere when there is need to convert document image to text.

% \subsection{Post office}


\section{Conclusion and Future enhancement}
In this work, we are presenting a comparative study of recognition based on supervised and 
unsupervised learning.  The former is based on SVM based training and classification(SVMTC) and the
later is based on Random Projection Technique(RPT).Overall, RPT gives better results than
SVMTC. 
The accuracy of recognition can be improved based on hierarchal classification. In order to deal 
with similar characters, nearest neighbor search can be carried out. At the word level, the recognition
can be improved by making use of a dictionary. Finally, all the associated scripts and datasets are made available at \url{https://github.com/pradeepmathesh/tocr} for future enhancement.
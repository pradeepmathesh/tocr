\addcontentsline{toc}{chapter}{Abstract}

\chapter*{Abstract}

%\{Optical Character Recognition (OCR)}


% \begin{abstract}
%\begin{center}
%Optical Character Recognition (OCR) for Tamil language\\[0.2in]
%\end{center}
% OCR is robust for many European languages but it is not the case for Indic languages. 
% So the main goal of our work is to increase the robustness especially for Tamil language. 
% We could use the existing OCR like Tesseract or Ocropus but we found that it is not suitable
% for Tamil language. Moreover, the algorithms used in these OCR systems are not start-of-the-art 
% so we want to apply the latest algorithms and find out how good is the result. 
% We are dissecting the entire OCR system right from preprocessing to postprocessing.
% Our objective is to use this OCR system to create parallel corpus for Machine Translation 
% and possibly creating electronic encyclopedia from existing old documents. 
% \end{abstract}
\begin{spacing}{1.5}
Optical Character Recognition (OCR) system is robust for Latin languages but it is not so for Indic languages. It is partly because of the number of character classes and close similarity of characters.

In this work, we are presenting a comparative study of recognition based on supervised and 
unsupervised learning. The former is based on SVM based training and classification (SVMTC) and the
later is based on Random Projection Technique (RPT). The challenging part of this project 
is feature  extraction. We have extracted features based on active contour model, 
character geometry, moment invariants, random projection and Gabor filter. It is a combination of 
statistical and visual features. Overall, RPT gives better results than
SVMTC. The result based on RPT is found out to be 58 \%.
\end{spacing}


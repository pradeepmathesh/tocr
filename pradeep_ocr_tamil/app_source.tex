\chapter{Appendix}

\section{SOURCE CODE}


\subsection{Preprocessing tool}
\begin{verbatim}
import Image
import string
import subprocess
import os
IMAGES = 1
def roll(image, delta):
"Roll an image sideways"
xsize, ysize = image.size
delta = delta % xsize
if delta == 0: return image
part1 = image.crop((0, 0, delta, ysize))
part2 = image.crop((delta, 0, xsize, ysize))
image.paste(part2, (0, 0, xsize-delta, ysize))
image.paste(part1, (xsize-delta, 0, xsize, ysize))
return image
def createblobs():

for i in xrange(IMAGES):

filename = "testimages/test"+str(i+1)+".tif" #Tamil
im1 = Image.open(filename)

x,y = im1.size



os.system('tesseract '+filename+' testimages batch.nochop makebox')
os.system('cat testimages.box | cut -d " " -f2-5 &> tuple')
f = open("tuple",'r')
boxList = f.readlines()
f.close()






count = 1
for box in boxList:
tmp = box.rstrip()
parts = string.split(tmp)

x0 = int(parts[0])
y0 = y - int(parts[3])
x1 = int(parts[2])
y1 = y - int(parts[1])

boxer = (x0,y0,x1,y1)
region = im1.crop(boxer)
size = (x1-x0,y1-y0)

im2 = Image.new("RGB",size)
im2.paste(region,(0,0)) #source --> destination(0,0)
filename = "test"+str(i+1)+"/"+str(count)+".png"


im2.save(filename, "PNG")
count = count + 1

def makedirs():
for i in xrange(IMAGES):
os.system('rm -rf '+' test'+str(i+1))
for i in xrange(IMAGES):
os.system('mkdir -p '+'test'+str(i+1))
def cleanup():
os.system('rm -f blobs/*')






makedirs()
cleanup()
print 'creating blobs'
createblobs()

\end{verbatim}


\subsection{Ground truth generating tool}
\begin{verbatim}
#coding=UTF-8
import pygtk
pygtk.require('2.0')
import gtk
import codecs
import os
class Train:
def close_application(self, widget, event, data=None):
l = len(self.file)
tim = l - self.count
for i in xrange(tim):
x = self.count
self.fout.write(self.file[x])
self.count = self.count + 1
self.fin.close()
self.fout.close()
os.system('mv output.report.tmp1 '+ self.source)
gtk.main_quit()
return False

def button_clicked(self, widget, data=None):

x = self.count
self.count = self.count + 1
if self.count > self.numIm:
self.file.close()
gtk.main_quit()
else:
w = self.text.get_text()
if w != '':

chop = self.file[x].split('\n')
st = chop[0] + ' ' + w
self.fout.write(st+'\n')
else:
self.fout.write(self.file[x])
self.label.set_text(self.file[self.count]) # turn off label temporarily

self.image.set_from_file(self.dirprefix+str(self.count+1)+'.png')

self.text.set_text('')
self.window.set_title(str(self.count+1)+'.png')



def setfocus(self,widget,event):
key = event.keyval
if key == gtk.keysyms.Left or key == gtk.keysyms.Return:

self.button.grab_focus()

def __init__(self):




self.dirprefix='/home/pradeep/Desktop/project/dataset/orderrnbin/'
self.dirprefix='/home/pradeep/Desktop/project/dataset/tam/test16_prop_bin/'
self.dirprefix2='/home/pradeep/Desktop/project/dataset/'

self.count = 0

self.source = self.dirprefix2+'map_16_4201'
self.fin = codecs.open(self.source,encoding='utf-8', mode='r')
self.fout = codecs.open('output.report.tmp1',encoding='utf-8', mode='w+')
self.file = list(self.fin)
if self.count > 0:
for i in xrange(self.count):
self.fout.write(self.file[i])
img = os.listdir(self.dirprefix)
self.numIm = len(img)
del img
self.window = gtk.Window(gtk.WINDOW_TOPLEVEL)
self.window.set_title(str(self.count+1)+'.png')
self.window.connect("delete_event", self.close_application)
self.window.set_border_width(10)
self.window.set_position(gtk.WIN_POS_CENTER)

self.hbox = gtk.HBox()
x = self.file[self.count]
self.label = gtk.Label(x)
self.hbox.pack_start(self.label)
self.label.show()
self.window.add(self.hbox)
self.image = gtk.Image()

self.image.set_from_file(self.dirprefix+str(self.count+1)+".png")
self.image.show()

self.button = gtk.Button()
self.button.add(self.image)
self.button.connect("clicked", self.button_clicked, "2")
self.button.show()
self.hbox.pack_start(self.button)
self.text = gtk.Entry()
self.text.set_width_chars(4)
self.hbox.pack_end(self.text)

self.text.show()
self.hbox.show()
self.window.connect("key-press-event", self.setfocus)
self.window.set_default_size(450,40)
self.window.show()
def main():
gtk.main()
return 0
File: /home/pradeep/class/work/thesis/imp.py Page 4 of 5
if __name__ == "__main__":
Train()
main()
\end{verbatim}


\subsection{Recognition engine}
\begin{verbatim}
#coding=UTF-8
import os
import math
import numpy as np
import Image
import codecs
import time

modeldirprefix='/home/pradeep/Desktop/project/dataset/Number2/';


datadirprefix='/home/pradeep/class/poems/leisure/project/dataset/tam/test16_prop_bin/';

start = time.time()
dict = {1:'ள', 2:'கி', 3:'ட', 4:'ளி', 5:'ப', 6:'க', 7:'ற்', 8:'ை', 9:'ல்', 11:'த', 12:'ே',14:'கு', 15:'ெ', 16:'இ', 17:'ர', 18:'று', 20:'ம்', 21:'எ', 22:'ன்', 25:'க்', 26:'ஃ', 27:'அ', 33:'ஸ்', 34:'ா', 36:'ழு', 41:'ரு', 42:'வி', 43:'த்', 45:'ஞ', 47:'ரி', 49:'லு', 56:'வ', 57:'ட்', 58:'ன', 64:'டு', 66:'றி', 67:'ல', 72:'ப்', 73:'ச', 74:'ம', 76:'ஸ', 77:'ர்', 84:'ஒ', 89:'ய', 91:'யி', 92:'ந', 96:'ஆ', 100:'ள்', 108:'ற', 111:'னு', 114:'வு', 125:'பி', 143:'உ', 164:'பு', 168:'ச்', 178:'ழி', 205:'தி', 209:'து', 211:'கூ', 213:'மு', 225:'நி', 229:'ந்', 234:'ஜி', 251:'ய்', 263:'ங்', 288:'ண்', 305:'டி', 352:'ளு', 388:'சி', 438:'யூ', 498:'ஏ', 520:'ழ', 572:'மி', 607:'ஸி', 609:'நீ', 610:'ழ்', 676:'ஜ',883:'வ்', 936:'ஷி', 939:'சூ', 973:'நூ', 1215:'யு', 1256:'மூ', 1287:'ஈ', 1369:'சு', 1486:'னி', 1712:'டீ', 1735:'ஹ', 1892:'ஐ', 2047:'மீ', 2100:'ஷ', 2140:'வீ', 2216:'ஜ்', 2407:'லி', 3333:'ஞ்', 3377:'தீ', 3381:'கீ', 3668:'ண'}
im_dim = 64
M = im_dim*im_dim; # size of image vector, i.e., number of pixels
d = 1000

charidx = os.listdir(modeldirprefix)
testidx = os.listdir(datadirprefix)



numIm = len(charidx)
numfiles = len(testidx)
DATA = np.mat(np.zeros((M,numIm),float))
j = 0

for i in charidx:
t = '%s%s' % (modeldirprefix,i);
im = Image.open( t ).convert('L')
im = im.resize((im_dim,im_dim))
m,n = im.size
x = np.mat(im)
DATA[:,j] = x.reshape(M,1) #assign vector to a particular of matrix
j = j + 1

R = np.random.rand(d,M);# using random matrix
phi = R*DATA
con = u''
fout = codecs.open('/home/pradeep/Desktop/project/dataset/output_test16', encoding='utf-8', mode='w+')
File: /home/pradeep/class/work/thesis/imp.py Page 5 of 5

for i in xrange(numfiles):



t = '%s%s.png' % (datadirprefix,str(i+1));


im1 = Image.open(t).convert('L') #if not in grayscale

im1 = im1.resize((im_dim,im_dim))
T_DATA = np.mat(im1)
T_DATA = T_DATA.reshape(M,1).copy()

test = R*T_DATA
test_n=test / math.sqrt(test.H*test)
norm2_phi = np.mat(np.zeros((len(phi),numIm),float))

for j in xrange(numIm):
norm2 = math.sqrt(phi[:,j].H*phi[:,j])
norm2_phi[:,j]= phi[:,j]/norm2
x = norm2_phi.H*test_n
index = charidx[x.argmax(0)]

split = index.split('.')


con = dict[int(split[0])].decode('utf-8')
fout.write(con+'\n')
fout.close()
end = time.time()
os.system('gedit /home/pradeep/Desktop/project/dataset/output_test16 &')
print 'It took %s seconds' % (end-start)
\end{verbatim}
\subsection{SVMTC}
\begin{verbatim}
function omi(omega);
	tic
	clc;
	%omega = 0; %0 means test!...  1 means train!
	path(path,'feature_extraction');
	path(path,'vowel_hu_moment');
	path(path,'aff_mom');
	path(path,'mom_inv');
	path(path,'Gabor');
	%dirprefix='/home/pradeep/Desktop/project/dataset/blob1250/';

	%training data location
	if(omega == 1)
	dirprefix='/home/pradeep/Desktop/project/dataset/tam/test16_prop_bin/';
	%dirprefix='/home/pradeep/Desktop/project/dataset/small/';
	fout = fopen('/home/pradeep/Desktop/project/dataset/feature_train.txt','w+'); % always overwrite
	else
	%testing data location
	dirprefix='/home/pradeep/Desktop/project/dataset/orderrnbin/';
	%dirprefix='/home/pradeep/Desktop/project/dataset/small/';
	fout = fopen('/home/pradeep/Desktop/project/dataset/feature_test.txt','w+'); % always overwrite
	end
	xdir = dir(dirprefix);
	numIm = size(xdir,1);
	fin=readinv('afinv12indep.txt'); 
	vstrsp = [];
	vmw = [];
	vrings = [];
	vmaxs_maxpoint = [];
	% Setting
	count = 1;
	ccount = 1;
	resize_im=32;
	M = resize_im*resize_im;  % size of image vector, i.e., number of pixels
	d = 100;   % dim of new feature space
	DATA = [];
	dillips_feature = [];
	humom_feature = [];
	for i = 1 : numIm
	%xdir(i).name
		if(strcmp(xdir(i).name,'.') || strcmp(xdir(i).name,'..'))
			continue;
		end
		MODEL_SMOOTH_IMAGE = 0; % Chan-Vese Model

		maxI = 256; % max image intensity

		t = sprintf('%s%s',dirprefix,xdir(i).name);
		Img = imread(t);
		[Ny,Nx,Nc] = size(Img); 
		if Nc>1; 
			Img=rgb2gray(Img); 
		end; % Convert color image to gray-scale image
	%	Img=imresize(Img, [32 32]); 
		ModelSeg = MODEL_SMOOTH_IMAGE;

		mu = 1e3; 
		lambda = mu* 5;
		cpt_fig = 1; % figure number
		VecParameters = [Ny; Nx; ModelSeg; lambda; mu; ]; % parameters for active contour function
		[u,c,h]=active_contour_minimization(Img,VecParameters,cpt_fig); % active contour function credit goes to Xavier bresson, UCLA

		x=c(2,:);
		L=length(x);
		count=1;
		ind=0;
		y=[];
		ct = [];
		while count < L
			ind=ind+1;
			y(ind)=x(count);
			ct(ind) = count;
			count=count+y(ind)+1;
		end
		[l,m] = max(y);
		w = length(y);
		if(m == w)
		  xc = c(1,ct(m)+1:L);
		  yc = c(2,ct(m)+1:L);
		else	  
		  xc = c(1,ct(m)+1:ct(m+1)-1);
		  yc = c(2,ct(m)+1:ct(m+1)-1);
		end
		minx = min(xc);
		maxx = max(xc);	
		miny = min(yc);
		maxy = max(yc);
		[dum,numContours] = size(y);
		[maxs_maxpoint,dummy] = max(y);
		rings = 0;
		for j = 1:numContours
			if j ~= numContours
				txc = c(1,ct(j)+1:ct(j+1)-1);
				tyc = c(2,ct(j)+1:ct(j+1)-1);
			else
				txc = c(1,ct(j)+1:L);
				tyc = c(2,ct(j)+1:L);
			end
			tminx = min(txc);
			tmaxx = max(txc);	
			tminy = min(tyc);
			tmaxy = max(tyc);
			if tminx > minx  && tmaxx < maxx
				rings = rings + 1;
			end 
		end
		w = size(y,2);
		s = xdir(i).name;
		[strsp dum] = strtok(s,'.');

		strsp = str2num(strsp);
		vstrsp = [vstrsp strsp];
		vmw = [vmw w];
		vrings = [vrings rings];
		vmaxs_maxpoint = [vmaxs_maxpoint maxs_maxpoint];
		Img=imresize(Img, [32 32]); 
		Img = im2double(Img);
		DATA(:, ccount) = reshape (Img, resize_im*resize_im, 1);%random projection
		v(ccount,:) = testgabor(Img); % gabor feature
		mm=cm(Img,12); % moment computation
		v1(ccount,:)=cafmi(fin,mm); % invariant evaluation
		% Geometry Invariant Properties based feature  ## credit goes to Dinesh Dillip, IISc 
		features=feature_extractor_2d(Img);
		dillips_feature = [dillips_feature features];
		% Hu's moment based features ## credit goes to Vivek Kwasara, UNC
		humom = [humoments(Img)]';
		humom_feature = [humom_feature 	humom];
		ccount = ccount + 1;
	end

	%Random Projection ## credit goes to Qinfeng Shi, University of Adelaide

	A = DATA;

	randn('state',200);
	R = randn(d,M);% using random matrix
	phi = R*A;

	for j = 1:size(phi,2)
		norm2 = norm(phi(:,j),2);
		norm2_phi(:,j)= phi(:,j)./norm2;
	end
	if(omega == 1)
	%else
		%load feature_train_essence.mat
	%	fprintf(fout,'%d\n%d\n',numIm-2,555);
		for i=1:numIm-2
			fprintf(fout,'%d ',vstrsp(i));
			fprintf(fout,'%f %f %f ',vmw(i),vrings(i),vmaxs_maxpoint(i));% stored as maxvmw, maxvmaxs_maxpoint,maxrings total number of points,number of points of maximum contour length,number of rings
			for k=1:d
				fprintf(fout,'%f ',norm2_phi(k,i));%stored
			end
			for j=1:55
				fprintf(fout,'%f ',dillips_feature(j,i));
			end
			for w=1:size(humom,1)
				fprintf(fout,'%f ',humom_feature(w,i));
			end
			for w=1:310
				fprintf(fout,'%f ',v(i,w));
			end
			for w=1:80
				fprintf(fout,'%f ',v1(i,w));
			end
			fprintf(fout,'\n');
		end
	else
		%save('feature_test.mat');
		for i=1:numIm-2
			fprintf(fout,'%d ',vstrsp(i));
			fprintf(fout,'%f %f %f ',vmw(i),vrings(i),vmaxs_maxpoint(i));% stored as maxvmw, maxvmaxs_maxpoint,maxrings total number of points,number of points of maximum contour length,number of rings
			for k=1:d
				fprintf(fout,'%f ',norm2_phi(k,i));%stored
			end
			for j=1:55
				fprintf(fout,'%f ',dillips_feature(j,i));
			end
			for w=1:size(humom,1)
				fprintf(fout,'%f ',humom_feature(w,i));
			end
			for w=1:310
				fprintf(fout,'%f ',v(i,w));
			end
			for w=1:80
				fprintf(fout,'%f ',v1(i,w));
			end
			fprintf(fout,'\n');
		end
	end
	fclose(fout);
	toc
\end{verbatim}